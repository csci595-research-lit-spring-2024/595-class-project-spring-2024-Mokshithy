\chapter{Introduction}
\label{ch:into} % This how you label a chapter and the key (e.g., ch:into) will be used to refer this chapter ``Introduction'' later in the report. 
% the key ``ch:into'' can be used with command \ref{ch:intor} to refere this Chapter.

\textbf{Guidance on introduction chapter writing:} Introductions are written in the following parts:
\begin{itemize}
    \item A brief  description of the investigated problem.
    \item A summary of the scope and context of the project, i.e., what is the background of the topic/problem/application/system/algorithm/experiment/research question/hypothesis/etc. under investigation/implementation/development [whichever is applicable to your project].
    \item The aims and objectives of the project.
    \item A description of the problem and the methodological approach adopted to solve the problem.
    \item A summary of the most significant outcomes and their interpretations.
    \item Organization of the report. 
\end{itemize}


Consult \textbf{your supervisor} to check the content of the introduction chapter. In this template, we only offer basic sections of an introduction chapter. It may  not be complete and comprehensive. Writing a report is a subjective matter, and a report's style and structure depend on the ``type of project'' as well as an individual's preference. This template suits the following project paradigms:
\begin{enumerate}
    \item software engineering and software/web application development;
    \item algorithm implementation, analysis and/or application;  
    \item science lab (experiment); and
    \item pure theoretical development (not mention extensively).
\end{enumerate}

Use only a single \textbf{font} for the body text. We recommend using a clean and electronic document friendly font like \textbf{Arial} or \textbf{Calibri} for MS-word (If you create a report in MS word). If you use this template, DO NOT ALTER the template's default font ``amsfont default computer modern''. The default \LaTeX~font ``computer modern'' is also acceptable. 

The recommended body text \textbf{font size} is minimum \textbf{11pt} and minimum one-half line spacing. The recommended figure/table caption font size is minimum 10pt. The footnote\footnote{Example footnote: footnotes are useful for adding external sources such as links as well as extra information on a topic or word or sentence. Use command \textbackslash footnote\{...\} next to a word to generate a footnote in \LaTeX.} font size is minimum 8pt. DO NOT ALTER the font setting of this template.   

%%%%%%%%%%%%%%%%%%%%%%%%%%%%%%%%%%%%%%%%%%%%%%%%%%%%%%%%%%%%%%%%%%%%%%%%%%%%%%%%%%%
\section{Background}
\label{sec:into_back}
% Describe to a reader the context of your project. That is, what is your project and what its motivation. Briefly explain the major theories, applications, and/or products/systems/algorithms whichever is relevant to your project.



The proficiency of Large Language Models (LLMs) like ChatGPT in generating working code is intrinsically linked to their understanding and application of data structures and algorithms (DSA). These fundamental components form the backbone of efficient and effective software development. DSA are critical in determining how data is organized, stored, and manipulated within a program, impacting everything from the execution speed to resource utilization. The ability of an AI model to adeptly handle these aspects is indicative of its depth in coding knowledge and its applicability in real-world software development scenarios.

In this context, the capabilities of ChatGPT in DSA are particularly noteworthy. This model has demonstrated a remarkable ability to not only understand and implement standard data structures and algorithms but also to apply them creatively to solve complex problems. The implication of this proficiency is significant; it suggests that LLMs like ChatGPT can be invaluable tools in the software development process, assisting in everything from initial problem analysis to the formulation of efficient algorithmic solutions.

Furthermore, the application of DSA in code generation by LLMs also opens up possibilities for more advanced software development applications. For instance, an AI model proficient in DSA can potentially assist in optimizing existing codebases, refactoring inefficient structures, or even suggesting algorithmic improvements. 

In the current landscape of AI-driven code generation, particularly with models like ChatGPT, there is a noticeable challenge in consistently generating code that is both efficient and correct. The generation of correct code is fundamental, as errors in code can lead to significant issues in software development, ranging from minor bugs to critical system failures. While ChatGPT exhibits a strong capacity for code generation, its performance can vary, especially in terms of efficiency and adherence to best practices in data structures and algorithms (DSA). This variability underscores the necessity for more refined techniques in interacting with these models to elicit the highest quality of code output.

Prompt engineering emerges as a crucial element in this context. It plays a critical role in harnessing their full potential, especially in code generation. It involves meticulously crafting input prompts to effectively communicate the requirements and constraints of a given problem to the AI model. This practice is particularly essential in programming scenarios where the quality of output is directly influenced by how the problem is presented to the model, i.e., the clarity and specificity of instructions can significantly influence the accuracy and utility of the generated code. A well-engineered prompt can lead to solutions that are not only syntactically and logically correct but also optimized in terms of performance and resource management. The significance of prompt design in obtaining optimal results from generative tasks has been highlighted in the research by \cite{chen2021evaluating} on GPT models.

The implications of a system proficient in accurately generating working, logical code are far-reaching. In the future, such capabilities could revolutionize software development, enabling faster deployment of robust and sophisticated applications. Furthermore, as noted by Chen et al. (2021) in their analysis of code generation models, these advancements could democratize programming, allowing individuals with limited coding expertise to develop software solutions through intuitive, natural language instructions. This could potentially lead to a surge in innovation, as barriers to software creation are lowered, and a broader range of perspectives are brought into the technology development process.

The current research aims to delve deeper into ChatGPT’s capabilities in DSA. It aims to assess and quantify the impact of prompt engineering on ChatGPT’s performance in generating algorithmic solutions. By evaluating the model’s performance in this domain, the study seeks to shed light on the extent to which ChatGPT can accurately and effectively generate code solutions that are not just correct in their logic, but also optimal in their use of data structures and algorithms. This exploration is pivotal, as it directly relates to the practical utility of LLMs in software engineering, a field where efficiency and optimization are paramount.




%%%%%%%%%%%%%%%%%%%%%%%%%%%%%%%%%%%%%%%%%%%%%%%%%%%%%%%%%%%%%%%%%%%%%%%%%%%%%%%%%%%
\section{Problem statement}
\label{sec:intro_prob_art}
% This section describes the investigated problem in detail. You can also have a separate chapter on ``Problem articulation.''  For some projects, you may have a section like ``Research question(s)'' or ``Research Hypothesis'' instead of a section on ``Problem statement.'
The central problem this study addresses is the assessment of the impact of prompt engineering on the performance of ChatGPT, specifically in solving undergraduate-level data structures and algorithms (DSA) questions. The effectiveness of Large Language Models (LLMs) like ChatGPT in code generation has been increasingly recognized. However, their ability to consistently produce efficient and correct solutions in the context of complex DSA problems remains an area requiring deeper exploration. This research hypothesizes that through strategic prompt engineering, the proficiency of ChatGPT in generating accurate and optimized solutions to DSA challenges can be significantly enhanced.

Prompt engineering, in this context, refers to the deliberate design and structuring of input prompts to guide ChatGPT in understanding and effectively responding to the intricacies of DSA problems. The hypothesis is grounded in the premise that the manner in which a problem is presented to an LLM can profoundly influence the quality of the generated solution. This hypothesis aligns with the findings of \cite{radford2019language}, who highlighted the importance of prompt design in achieving desired outcomes from generative AI models. The study aims to empirically test this hypothesis by systematically varying the prompts used to interact with ChatGPT and evaluating the resulting code's correctness, efficiency, and adherence to DSA best practices.

The significance of this investigation lies not only in its potential to enhance the understanding of prompt engineering as a tool for optimizing LLM performance but also in its broader implications for the field of AI-driven software development. By establishing a clear correlation between prompt engineering and the quality of ChatGPT-generated code, the study seeks to contribute to the development of more effective methodologies for leveraging LLMs in technical and educational settings.



%%%%%%%%%%%%%%%%%%%%%%%%%%%%%%%%%%%%%%%%%%%%%%%%%%%%%%%%%%%%%%%%%%%%%%%%%%%%%%%%%%%
\section{Aims and objectives}
\label{sec:intro_aims_obj}
Describe the ``aims and objectives'' of your project. 

\textbf{Aims:} The aims tell a read what you want/hope to achieve at the end of the project. The  aims define your intent/purpose in general terms.  

\textbf{Objectives:} The objectives are a set of tasks you would perform in order to achieve the defined aims. The objective statements have to be specific and measurable through the results and outcome of the project.



%%%%%%%%%%%%%%%%%%%%%%%%%%%%%%%%%%%%%%%%%%%%%%%%%%%%%%%%%%%%%%%%%%%%%%%%%%%%%%%%%%%
\section{Solution approach}
\label{sec:intro_sol} % label of Org section
Briefly describe the solution approach and the methodology applied in solving the set aims and objectives.

Depending on the project, you may like to alter the ``heading'' of this section. Check with you supervisor. Also, check what subsection or any other section that can be added in or removed from this template.

\subsection{A subsection 1}
\label{sec:intro_some_sub1}
You may or may not need subsections here. Depending on your project's needs, add two or more subsection(s). A section takes at least two subsections. 

\subsection{A subsection 2}
\label{sec:intro_some_sub2}
Depending on your project's needs, add more section(s) and subsection(s).

\subsubsection{A subsection 1 of a subsection}
\label{sec:intro_some_subsub1}
The command \textbackslash subsubsection\{\} creates a paragraph heading in \LaTeX.

\subsubsection{A subsection 2 of a subsection}
\label{sec:intro_some_subsub2}
Write your text here...

%%%%%%%%%%%%%%%%%%%%%%%%%%%%%%%%%%%%%%%%%%%%%%%%%%%%%%%%%%%%%%%%%%%%%%%%%%%%%%%%%%%
\section{Summary of contributions and achievements} %  use this section 
\label{sec:intro_sum_results} % label of summary of results
Describe clearly what you have done/created/achieved and what the major results and their implications are. 


%%%%%%%%%%%%%%%%%%%%%%%%%%%%%%%%%%%%%%%%%%%%%%%%%%%%%%%%%%%%%%%%%%%%%%%%%%%%%%%%%%%
\section{Organization of the report} %  use this section
\label{sec:intro_org} % label of Org section
Describe the outline of the rest of the report here. Let the reader know what to expect ahead in the report. Describe how you have organized your report. 

\textbf{Example: how to refer a chapter, section, subsection}. This report is organised into seven chapters. Chapter~\ref{ch:lit_rev} details the literature review of this project. In Section~\ref{ch:method}...  % and so on.

\textbf{Note:}  Take care of the word like ``Chapter,'' ``Section,'' ``Figure'' etc. before the \LaTeX command \textbackslash ref\{\}. Otherwise, a  sentence will be confusing. For example, In \ref{ch:lit_rev} literature review is described. In this sentence, the word ``Chapter'' is missing. Therefore, a reader would not know whether 2 is for a Chapter or a Section or a Figure.

