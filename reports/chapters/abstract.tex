%Two resources useful for abstract writing.
% Guidance of how to write an abstract/summary provided by Nature: https://cbs.umn.edu/sites/cbs.umn.edu/files/public/downloads/Annotated_Nature_abstract.pdf %https://writingcenter.gmu.edu/guides/writing-an-abstract
\chapter*{\center \Large  Abstract}
%%%%%%%%%%%%%%%%%%%%%%%%%%%%%%%%%%%%%%
% Replace all text with your text
%%%%%%%%%%%%%%%%%%%%%%%%%%%%%%%%%%%

The realm of Artificial Intelligence (AI) is undergoing a significant transformation, primarily 
driven by the advancements in Large Language Models (LLMs). Among these, ChatGPT has emerged as 
a standout model, acclaimed for its exceptional ability in conducting multi-turn conversations 
and showcasing coding proficiency in various programming languages. This study deals with the 
investigation of the performance of ChatGPT and the impact of prompt engineering, on its 
effectiveness in solving standardized undergraduate-level data structures and algorithms 
problems. A novel aspect of this research is the focus on automating the entire evaluation 
pipeline, including prompt fine-tuning, generating responses from ChatGPT, and systematically 
testing these responses against a curated set of standard questions. This automation not only 
streamlines the assessment process but also sets a precedent for analyzing other LLMs in a 
similar fashion.

The methodology centers on the development and application of tailored prompts designed to 
maximize ChatGPT's performance in solving complex programming challenges. The study 
meticulously curates a diverse collection of data structures and algorithms questions, 
representative of undergraduate coursework. ChatGPT's responses to these prompts are then 
automatically processed and evaluated against multiple test cases to determine their 
correctness and efficacy.

Key findings of this research will illuminate the potential of prompt engineering as a crucial 
factor in enhancing the performance of LLMs in technical domains. The outcomes are expected to 
provide valuable insights into the capabilities and limitations of ChatGPT in the context of 
algorithmic problem-solving. Furthermore, the study's automated approach promises scalability 
and reproducibility, offering a robust framework for future research in LLM performance 
analysis across various disciplines.


%%%%%%%%%%%%%%%%%%%%%%%%%%%%%%%%%%%%%%%%%%%%%%%%%%%%%%%%%%%%%%%%%%%%%%%%%s
~\\[1cm]
\noindent % Provide your key words
\textbf{Keywords:} ChatGPT, Large Language Models, Prompt Engineering, Data Structures, Algorithms.

\vfill
\noindent
% \textbf{Report's total word count:} we expect a maximum of 10,000 words (excluding reference and appendices) and about 10 pages. [A good project report can also be written in approximately 5,000 words.]

