% \chapter{Reflection}
% \label{ch:reflection}
% %%%%%%%%%%%%%%%%%%%%%%%%%%%%%%%
% %% Please remove/replace text below
% %%%%%%%%%%%%%%%%%%%%%%%%%%%%%%%
% Write a short paragraph on the substantial learning experience. This can include your decision-making approach in problem-solving.

% \textbf{Some hints:} You obviously learned how to use different programming languages, write reports in \LaTeX and use other technical tools. In this section, we are more interested in what you thought about the experience. Take some time to think and reflect on your individual project as an experience, rather than just a list of technical skills and knowledge. You may describe things you have learned from the research approach and strategy, the process of identifying and solving a problem, the process research inquiry, and the understanding of the impact of the project on your learning experience and future work.

% Also think in terms of:
% \begin{itemize}
%     \item what knowledge and skills you have developed
%     \item what challenges you faced, but was not able to overcome
%     \item what you could do this project differently if the same or similar problem would come
%     \item rationalize the divisions from your initial planed aims and objectives.
% \end{itemize}


% A good reflective summary could be approximately 300--500 words long, but this is just a recommendation.

% ~\\[2em]
% \noindent
% {\huge \textbf{Note:}} The next chapter is ``\textbf{References},'' which will be automatically generated if you are using BibTeX referencing method. This template uses BibTeX referencing.  Also, note that there is difference between ``References'' and ``Bibliography.'' The list of ``References'' strictly only contain the list of articles, paper, and content you have cited (i.e., refereed) in the report. Whereas Bibliography is a list that contains the list of articles, paper, and content you have cited in the report plus the list of articles, paper, and content you have read in order to gain knowledge from. We recommend to use only the list of ``References.'' 
\chapter{Reflection}
\label{ch:reflection}

This project has been a profound journey into the capabilities and limitations of artificial intelligence, specifically within the context of solving data structures and algorithms (DSA) problems using ChatGPT. Reflecting on this experience, I recognize it as not only a technical exploration but also as a significant period of personal and professional growth.

One of the initial challenges I faced was the lack of libraries or open-source resources specifically tailored for gathering coding problems and their test cases. This limitation necessitated a deep dive into existing methodologies, which often fell short in directly addressing the needs of this project. The absence of a ready-made solution for fetching diverse and complex coding questions led me to develop a new approach for data acquisition. This process involved exploring various databases and APIs, eventually leading to the creation of an automated pipeline that utilized the LeetCode GraphQL API. This experience was invaluable as it pushed me to innovate and create tools that were not only useful for this project but also adaptable for future research in this field.

Throughout this project, I learned the critical importance of role-based prompt engineering. By crafting prompts that simulated different user roles—from a junior programmer to an expert in Python programming—I was able to significantly influence ChatGPT's output. This exploration revealed how subtle changes in language and the inclusion of specific keywords could dramatically alter the effectiveness of the solutions provided by the AI. The experience has sharpened my skills in understanding how to guide AI to perform tasks that it might not inherently excel at, especially in areas requiring a deep understanding of complex data structures.

A substantial portion of my time was devoted to analyzing the responses generated by ChatGPT, identifying patterns, and fine-tuning the post-processing of these solutions to ensure they could be executed on test cases without human intervention. This aspect of the project was particularly challenging due to the variability in the AI's responses. Developing a methodical approach to automatically adjust and correct the generated code was both challenging and rewarding. It underscored the importance of iterative testing and adjustment in AI-related projects, where outcomes can be unpredictable and require continuous refinement.

Looking back, if I were to tackle a similar problem in the future, I would place a greater emphasis on developing even more robust tools for data handling and analysis from the outset. The iterative nature of AI testing and the variability in model performance highlighted the need for flexible and adaptable project planning. 

Moreover, this project has significantly impacted my understanding of the role of AI in educational and professional settings. It has provided me with a solid foundation in research methodology, enhanced my problem-solving skills, and deepened my appreciation for the detailed and careful design required to successfully implement AI solutions. The insights gained from this project will undoubtedly influence my approach to future AI research and applications, particularly in optimizing the interaction between human input and AI output.

Overall, this reflective summary encapsulates not just a list of technical skills acquired but also the broader learning experiences—highlighting challenges, adaptations, and personal insights gained throughout the duration of the project.
