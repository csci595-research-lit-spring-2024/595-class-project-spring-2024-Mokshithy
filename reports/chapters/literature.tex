\chapter{Literature Review}
\label{ch:lit_rev} %Label of the chapter lit rev. The key ``ch:lit_rev'' can be used with command \ref{ch:lit_rev} to refer this Chapter.

% A literature review chapter can be organized in a few sections with appropriate titles. A literature review chapter might  contain the following:
% \begin{enumerate}
%     \item A review of the state-of-the-art (include theories and solutions) of the field of research.
%     \item A description of the project in the context of existing literature and products/systems.
%     \item An analysis of how the review is relevant to the intended application/system/problem.
%     \item A critique of existing work compared with the intended work.
% \end{enumerate}
% Note that your literature review should demonstrate the significance of the project.

% % PLEAE CHANGE THE TITLE of this section
% \section{Example of in-text citation of references in \LaTeX} 
% % Note the use of \cite{} and \citep{}
% The references in a report relate your content with the relevant sources, papers, and the works of others. To include references in a report, we \textit{cite} them in the texts. In MS-Word, EndNote, or MS-Word references, or plain text as a list can be used. Similarly, in \LaTeX, you can use the ``thebibliography'' environment, which is similar to the plain text as a list arrangement like the MS word. However, In \LaTeX, the most convenient way is to use the BibTex, which takes the references in a particular format [see references.bib file of this template] and lists them in style [APA, Harvard, etc.] as we want with the help of proper packages.    

% These are the examples of how to \textit{cite} external sources, seminal works, and research papers. In \LaTeX, if you use ``\textbf{BibTex}'' you do not have to worry much since the proper use of a bibliographystyle package like ``agsm for the Harvard style'' and little rectification of the content in a BiBText source file [In this template, BibTex are stored in the ``references.bib'' file], we can conveniently generate  a reference style. 

% Take a note of the commands \textbackslash cite\{\} and \textbackslash citep\{\}. The command \textbackslash cite\{\} will write like ``Author et al. (2019)'' style for Harvard, APA and Chicago style. The command \textbackslash citep\{\} will write like ``(Author et al., 2019).'' Depending on how you construct a sentence, you need to use them smartly. Check the examples of \textbf{in-text citation} of sources listed here [This template recommends the \textbf{Harvard style} of referencing.]:
% \begin{itemize}
%     \item \cite{lamport1994latex} has written a comprehensive guide on writing in \LaTeX ~[Example of \textbackslash cite\{\} ].
%     \item If \LaTeX~is used efficiently and effectively, it helps in writing a very high-quality project report~\citep{lamport1994latex} ~[Example of \textbackslash citep\{\} ].   
%     \item A detailed APA, Harvard, and Chicago referencing style guide are available in~\citep{uor_refernce_style}.
% \end{itemize}

% \noindent 
% Example of a numbered list:
% \begin{enumerate}
%     \item \cite{lamport1994latex} has written a comprehensive guide on writing in \LaTeX.
%     \item If \LaTeX is used efficiently and effectively, it helps in writing a very high-quality project report~\citep{lamport1994latex}.   
% \end{enumerate}

% % PLEAE CHANGE THE TITLE of this section
% \section{Example of ``risk'' of unintentional plagiarism}
% Using other sources, ideas, and material always bring with it a risk of unintentional plagiarism. 

% \noindent
% \textbf{\color{red}MUST}: do read the university guidelines on the definition of plagiarism as well as the guidelines on how to avoid plagiarism~\citep{uor_plagiarism}.

Our investigation delves into two primary areas of research concerning ChatGPT: its coding capabilities and its performance in non-coding tasks. This review delineates these areas and positions our study within the broader research landscape.

Previous studies have extensively explored ChatGPT's coding skills. Notably, Noever and McKee ([Noever et al.](https://arxiv.org/abs/2301.13382)) evaluated ChatGPT's proficiency in performing CRUD operations using datasets like Iris, Titanic, Boston Housing, and Faker. Their findings indicated that ChatGPT could effectively emulate a Python interpreter, generating code and outputs autonomously. This demonstrates ChatGPT's capability to handle structured data and execute CRUD operations efficiently. Biswas et al. ([Biswas et al.](https://doi.org/10.58496/MJCSC/2023/002)) further explored ChatGPT's role as a programming aide, highlighting its assistance in code completion, debugging, and refactoring tasks.

A study paralleling our methodology ([Bubeck et al.](https://arxiv.org/abs/2303.12712)) tasked ChatGPT with creating Python functions, initially using the HumanEval dataset and later a set of LeetCode problems. This research benchmarked GPT-4 against other models and human performance, revealing GPT-4's superior capabilities. Another comprehensive analysis ([Tian et al.](https://arxiv.org/abs/2304.11938)) examined ChatGPT's code translation, correction, and comprehension skills, offering a nuanced view of its coding proficiency but also noting occasional inaccuracies and inefficiencies.

Moreover, a study by Feng et al. ([Feng et al.](https://arxiv.org/abs/2304.11938)) employed a crowdsourcing approach to evaluate ChatGPT's code generation capabilities, analyzing social media discourse to glean insights into programming language preferences, usage scenarios, and common errors in generated code snippets.

Our research aims to extend these findings by focusing on a broader set of coding challenges and evaluating ChatGPT's coding quality across various programming tasks, particularly in data structures and algorithms. This approach allows us to contribute novel insights into ChatGPT's applicability and efficiency in software development.

The application of ChatGPT extends beyond coding. Gilson et al. ([Gilson et al.](https://doi.org/10.2196/45312)) assessed ChatGPT's ability to answer medical licensing exam questions, noting its proficiency in recall questions but limitations in reasoning. Similarly, studies on ChatGPT's mathematical skills ([Frieder et al.](https://arxiv.org/abs/2301.13867); [Pardos and Bhandari](https://arxiv.org/abs/2302.06871)) revealed its creative reasoning capabilities but highlighted deficiencies in critical reasoning and technical accuracy.

Additionally, research by Zhuo et al. ([Zhuo et al.](https://arxiv.org/abs/2301.12867)) and Bang et al. ([Bang et al.](https://arxiv.org/abs/2302.04023)) investigated ChatGPT's linguistic abilities and ethical implications, uncovering challenges related to bias, reliability, and understanding of non-Latin scripts.

While these studies offer valuable perspectives on ChatGPT's utility in diverse domains, our research is distinctly focused on optimizing ChatGPT's performance in solving data structures and algorithms problems through strategic prompt engineering. By focusing on this niche, we aim to enhance the understanding of how strategic prompt design can optimize ChatGPT's performance in technical education and software development. By examining the impact of prompt engineering on ChatGPT's performance, our research seeks to bridge gaps in the current literature and contributes to the field by exploring methods to harness ChatGPT's potential more effectively, particularly in areas requiring sophisticated algorithmic thinking and problem-solving skills.






% A possible section of you chapter
\section{Critique of the review} % Use this section title or choose a betterone
Describe your main findings and evaluation of the literature. ~\\

% Pleae use this section
\section{Summary} 
Write a summary of this chapter~\\
