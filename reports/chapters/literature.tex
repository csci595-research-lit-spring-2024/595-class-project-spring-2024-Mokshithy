\chapter{Literature Review}
\label{ch:lit_rev} %Label of the chapter lit rev. The key ``ch:lit_rev'' can be used with command \ref{ch:lit_rev} to refer this Chapter.

% A literature review chapter can be organized in a few sections with appropriate titles. A literature review chapter might  contain the following:
% \begin{enumerate}
%     \item A review of the state-of-the-art (include theories and solutions) of the field of research.
%     \item A description of the project in the context of existing literature and products/systems.
%     \item An analysis of how the review is relevant to the intended application/system/problem.
%     \item A critique of existing work compared with the intended work.
% \end{enumerate}
% Note that your literature review should demonstrate the significance of the project.

% % PLEAE CHANGE THE TITLE of this section
% \section{Example of in-text citation of references in \LaTeX} 
% % Note the use of \cite{} and \citep{}
% The references in a report relate your content with the relevant sources, papers, and the works of others. To include references in a report, we \textit{cite} them in the texts. In MS-Word, EndNote, or MS-Word references, or plain text as a list can be used. Similarly, in \LaTeX, you can use the ``thebibliography'' environment, which is similar to the plain text as a list arrangement like the MS word. However, In \LaTeX, the most convenient way is to use the BibTex, which takes the references in a particular format [see references.bib file of this template] and lists them in style [APA, Harvard, etc.] as we want with the help of proper packages.    

% These are the examples of how to \textit{cite} external sources, seminal works, and research papers. In \LaTeX, if you use ``\textbf{BibTex}'' you do not have to worry much since the proper use of a bibliographystyle package like ``agsm for the Harvard style'' and little rectification of the content in a BiBText source file [In this template, BibTex are stored in the ``references.bib'' file], we can conveniently generate  a reference style. 

% Take a note of the commands \textbackslash cite\{\} and \textbackslash citep\{\}. The command \textbackslash cite\{\} will write like ``Author et al. (2019)'' style for Harvard, APA and Chicago style. The command \textbackslash citep\{\} will write like ``(Author et al., 2019).'' Depending on how you construct a sentence, you need to use them smartly. Check the examples of \textbf{in-text citation} of sources listed here [This template recommends the \textbf{Harvard style} of referencing.]:
% \begin{itemize}
%     \item \cite{lamport1994latex} has written a comprehensive guide on writing in \LaTeX ~[Example of \textbackslash cite\{\} ].
%     \item If \LaTeX~is used efficiently and effectively, it helps in writing a very high-quality project report~\citep{lamport1994latex} ~[Example of \textbackslash citep\{\} ].   
%     \item A detailed APA, Harvard, and Chicago referencing style guide are available in~\citep{uor_refernce_style}.
% \end{itemize}

% \noindent 
% Example of a numbered list:
% \begin{enumerate}
%     \item \cite{lamport1994latex} has written a comprehensive guide on writing in \LaTeX.
%     \item If \LaTeX is used efficiently and effectively, it helps in writing a very high-quality project report~\citep{lamport1994latex}.   
% \end{enumerate}

% % PLEAE CHANGE THE TITLE of this section
% \section{Example of ``risk'' of unintentional plagiarism}
% Using other sources, ideas, and material always bring with it a risk of unintentional plagiarism. 

% \noindent
% \textbf{\color{red}MUST}: do read the university guidelines on the definition of plagiarism as well as the guidelines on how to avoid plagiarism~\citep{uor_plagiarism}.

In the rapidly evolving field of artificial intelligence, the exploration of Large Language Models (LLMs) like ChatGPT has become a focal point for researchers aiming to harness their capabilities for both coding and non-coding applications. Our study situates itself within this dynamic research landscape, specifically investigating ChatGPT's proficiency in data structures and algorithms (DSA)—a core component of computer science that underpins efficient software development. This literature review synthesizes pivotal findings from leading studies, setting the stage for our inquiry into how strategic prompt engineering can refine ChatGPT's DSA problem-solving abilities.
\subsection{ChatGPT's Coding Capabilities}
\subsubsection{CRUD Operations and Programming Assistance}

\cite{noever2023numeracy}'s seminal work offers an insightful evaluation of ChatGPT's ability to execute CRUD operations across several well-known datasets, revealing the model's adeptness at mimicking Python interpreter functionalities. This capacity for autonomous code generation and execution marks a significant stride in AI's integration into software development, particularly in handling structured data—a critical aspect our research aims to build upon by examining ChatGPT's effectiveness in more complex coding scenarios inherent in DSA challenges.

\cite{Biswas} extend the discussion on ChatGPT's utility as a programming assistant, underscoring its potential to streamline code completion, debugging, and refactoring tasks. This study illuminates the model's capacity to alleviate the manual burden associated with these activities, emphasizing its role as a facilitator of more efficient coding practices. Our research echoes this theme, delving into how ChatGPT can be further optimized through prompt engineering to tackle the nuanced demands of DSA problem-solving.
\subsubsection{Advanced Coding Skills}

The work of \cite{bubeck2023sparks} aligns closely with our methodological approach, challenging ChatGPT to develop Python functions using diverse problem sets. Their findings not only highlight GPT-4's superior performance but also identify critical areas for improvement when dealing with intricate coding tasks. This insight is particularly relevant to our study, which seeks to elucidate how varying prompt designs might enhance ChatGPT's ability to generate more accurate and efficient DSA solutions.

\cite{tian2023chatgpt}'s comprehensive analysis further details ChatGPT's capabilities in code translation, correction, and comprehension. While acknowledging the model's sophisticated coding proficiency, the study also points to its occasional inaccuracies and inefficiencies—challenges our research addresses by investigating the potential of prompt engineering to mitigate such issues in the context of DSA problems.

The innovative study by \cite{10196869} utilizes a crowdsourcing approach to assess ChatGPT's code generation capabilities. By analyzing public discourse on social media, they provide a nuanced understanding of common programming languages, usage scenarios, and prevalent errors in code snippets generated by ChatGPT. This exploration of public perception offers a backdrop against which our research examines the specificity and clarity of prompts in refining ChatGPT's output for DSA tasks.
\subsection{ChatGPT's Performance in Non-Coding Domains}
\subsubsection{Medical and Mathematical Skills}

The research conducted by \cite{info:doi/10.2196/45312} evaluates ChatGPT's capacity to respond to medical licensing exam questions, revealing its proficiency in recall-based queries but limitations in complex reasoning tasks. This delineation between ChatGPT's strengths and weaknesses in processing medical data provides a parallel to our study's focus on the model's problem-solving strategies within DSA, particularly how prompt engineering can address gaps in reasoning and technical accuracy.

Studies by \cite{frieder2023mathematical} delve into ChatGPT's mathematical abilities, highlighting its potential for creative reasoning alongside notable deficiencies in critical thinking and precision. These findings underscore the broader challenge of equipping ChatGPT with the skills to navigate the complexities of DSA problems effectively—a core aim of our investigation.
Linguistic Abilities and Ethical Considerations

The inquiries by \cite{zhuo2023red} into ChatGPT's linguistic capabilities and ethical dimensions reveal concerns over bias, reliability, and the model's grasp of non-Latin scripts. These studies contribute to a critical discourse on the responsible development and deployment of LLMs, framing our research within a broader conversation about the ethical use of ChatGPT in educational and software development settings.

In sum, our literature review not only highlights the significant strides made in understanding and utilizing ChatGPT's capabilities but also identifies the gaps and challenges that persist, particularly in the realm of DSA. Our research seeks to bridge these gaps through a focused examination of prompt engineering as a tool for enhancing ChatGPT's problem-solving efficacy. By exploring the nuanced interaction between prompt specificity and model output, our study contributes to the ongoing dialogue on leveraging ChatGPT for more sophisticated and efficient software development processes.




% A possible section of you chapter
\section{Critique of the review} % Use this section title or choose a betterone
Our main findings from the literature review reveal that while significant strides have been made in understanding and applying ChatGPT in both coding and non-coding contexts, several key areas require further exploration and development. Studies to date have effectively highlighted ChatGPT's capabilities and limitations, providing a solid foundation for assessing its applicability and performance in real-world tasks. However, a notable gap exists in the form of a lack of an automated pipeline to systematically test ChatGPT’s efficacy on data structures and algorithms (DSA) coding questions.

Current research predominantly examines ChatGPT's performance through isolated demonstrations of its capabilities, lacking a comprehensive framework that would allow for ongoing evaluation and refinement across a range of DSA challenges. This gap points to the need for more structured research efforts that not only test but also refine ChatGPT’s ability to handle complex coding problems. Moreover, the literature reveals that there is minimal focus on the development and testing of role-based prompts that could significantly enhance the model’s utility by tailoring its responses to specific user roles or coding scenarios. This type of prompt engineering could increase the relevance and precision of ChatGPT’s outputs, thereby improving its practical effectiveness in diverse programming environments.

This critique underscores the importance of advancing research to fill these gaps, particularly through the creation of an automated testing pipeline and the exploration of role-based prompt efficacy. Addressing these issues will be crucial for optimizing ChatGPT’s performance and ensuring its readiness for more sophisticated and varied applications in the field of software development. Such advancements could dramatically enhance the model's functional utility, making it a more powerful tool in the arsenal of programmers and developers facing complex DSA challenges. ~\\

% Pleae use this section
\section{Summary} 
This chapter has underscored the significant progress in the deployment of Large Language Models like ChatGPT across diverse domains, with a particular focus on their application in software development and problem-solving within the realm of data structures and algorithms. The literature reviewed showcases both the strengths and limitations of ChatGPT, from basic CRUD operations to more complex coding tasks requiring nuanced understanding and creativity. Our analysis reveals that while ChatGPT exhibits promising capabilities, there are substantial areas for improvement, particularly in enhancing the model's precision and reliability through prompt engineering.

Furthermore, the absence of an automated testing pipeline for continuous evaluation and the potential for role-based prompts suggest promising directions for future research. By developing methods to systematically assess and refine ChatGPT's responses to DSA challenges, researchers can better harness its capabilities for efficient problem-solving. Ultimately, our study contributes to the ongoing discussion on improving and leveraging AI tools like ChatGPT, aiming to create more sophisticated and effective solutions in the field of software development.~\\
